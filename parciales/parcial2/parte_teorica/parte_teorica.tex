% article example for classicthesis.sty
\documentclass[10pt,a4paper]{article} % KOMA-Script article scrartcl
\usepackage{import}
\usepackage{xifthen}
\usepackage{pdfpages}
\usepackage{transparent}
\newcommand{\incfig}[1]{%
    \def\svgwidth{\columnwidth}
    \import{./figures/}{#1.pdf_tex}
}
\usepackage{lipsum}     %lorem ipsum text
\usepackage{titlesec}   %Section settings
\usepackage{titling}    %Title settings
\usepackage[margin=10em]{geometry}  %Adjusting margins
\usepackage{setspace}
\usepackage{listings}
\usepackage{amsmath}    %Display equations options
\usepackage{amssymb}    %More symbols
\usepackage{xcolor}     %Color settings
\usepackage{pagecolor}
\usepackage{mdframed}
\usepackage[spanish]{babel}
\usepackage[utf8]{inputenc}
\usepackage{longtable}
\usepackage{multicol}
\usepackage{graphicx}
\graphicspath{ {./Images/} }
\setlength{\columnsep}{1cm}

% ====| color de la pagina y del fondo |==== %
\pagecolor{white}
\color{black}



\begin{document}
    %========================{TITLE}====================%
    \title{{  Parcial Desarrollo web Parte Teórica  }}
    \author{{Rodrigo Castillo}}
    \date{\today}

    \maketitle


    %=======================NOTES GOES HERE===================%
    \section{Describe brevemente cuál es la ventaja de utilizar web components
    en nuestros proyectos y cuál es su relación con los frameworks actuales.}

        Los componentes web son etiquetas personalizadas que nos permiten
        desarrollar de manera mas ágil y ligera, los frameworks actuales como
        Angular nos proporcionan una lista de web components previamente
        elaborados con distintas funcionalidades, con eso, ya no es necesario
        elaborar todo el diseño de un elemento principal de nuesta página sino
        que simplemente podemos modificar uno ya previamente hecho.

    \section{Describe brevemente que dignifica que javascript funcione de manera asíncrona}

        Como programadores estamos acostumbrados a que nuestros programas de
        ejecuten de manera secuencial (en lenguajes como Python, C++, Bash,
        Ruby, Java...)a menos que les indiquemos lo contrario. Esto ocurre
        porque los procesadores y el lenguaje de ensamblador funciona de manera
        secuencial,
        sin embargo, en el mundo del desarrollo web, los scripts se ejecutan de
        manera asincrónica pues es necesario que el browser no se sobrecargue y
        pueda cargar todo el contenido de la aplicación lo más rápido posible ,
        esto quiere decir que los algoritmos que están
        plasmados en el código no se ejecutan de manera secuencial.  esto puede
        conducir a problemas, bugs o vulnerabilidades al momento de desarrollar
        una aplicación si no se hace un manejo de esto de forma adecuada, puede
        resultar un dolor de cabeza para el desarrollador.
















    %=======================NOTES ENDS HERE===================%

    % bib stuff
    \nocite{*}
    \addtocontents{toc}{{}}
    \addcontentsline{toc}{section}{\refname}
    \bibliographystyle{plain}
    \bibliography{../Bibliography}
\end{document}

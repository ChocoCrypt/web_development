% article example for classicthesis.sty
\documentclass[10pt,a4paper]{article} % KOMA-Script article scrartcl
\usepackage{import}
\usepackage{xifthen}
\usepackage{pdfpages}
\usepackage{transparent}
\newcommand{\incfig}[1]{%
    \def\svgwidth{\columnwidth}
    \import{./figures/}{#1.pdf_tex}
}
\usepackage{lipsum}     %lorem ipsum text
\usepackage{titlesec}   %Section settings
\usepackage{titling}    %Title settings
\usepackage[margin=10em]{geometry}  %Adjusting margins
\usepackage{setspace}
\usepackage{listings}
\usepackage{amsmath}    %Display equations options
\usepackage{amssymb}    %More symbols
\usepackage{xcolor}     %Color settings
\usepackage{pagecolor}
\usepackage{mdframed}
\usepackage[spanish]{babel}
\usepackage[utf8]{inputenc}
\usepackage{longtable}
\usepackage{multicol}
\usepackage{graphicx}
\graphicspath{ {./Images/} }
\setlength{\columnsep}{1cm}

% ====| color de la pagina y del fondo |==== %
\pagecolor{white}
\color{black}



\begin{document}
    %========================{TITLE}====================%
    \title{{  Punto 1 Parcial  }}
    \author{{Rodrigo Castillo}}
    \date{\today}

    \maketitle


    %=======================NOTES GOES HERE===================%
    \section{Cuales son los diferentes usuarios que debería tener la aplicación?}
        \begin{enumerate}
            \item {investigadores}
            \item {voluntarios}
        \end{enumerate}

    \section{Escriba 4 historias de usuario adecuadas para este caso}
        \begin{itemize}
            \item {yo como investigador quiero postear casos anónimos para
                investigarlos posteriormente}
            \item {yo como usuario del grupo 1 quiero votar por soluciones para
                aconsejar a la gente}
            \item {yo como usuario del grupo 2 quiero votar entre opciones y
                darle me gusta para consejara la gente}
            \item {yo como usuario del grupo 3 quiero votar entre opciones o
                añadir una nueva para aconsejar}
        \end{itemize}


    \section{Realice un listado de los requerimientos funcionales y no funcionales de la aplicación según lo
    manifestado por el investigador.}
        requerimientos funcionales...
        \begin{itemize}
            \item {la aplicación debe almacenar los datos de los usuarios , nombre, alias, correo, genero y asignarle un grupo}
            \item {la aplicación deberá recopilar los datos de los post para posteriormente analizarlos}
            \item {los investigadores tienen que ser anónimos}
            \item {los consejos deben ser lo mas genuinos posibles}
        \end{itemize}
        requerimientos no funcionales
        \begin{itemize}
            \item {las opciones de post deben ser amigables para los usuarios}
            \item {los voluntarios deben sentirse cómodos al momento de postear}
            \item {no se debe permitir a los usuarios que pertenezcan al grupo
                3 y 4 a dar sugerencias negativas como consejos}
        \end{itemize}






















    %=======================NOTES ENDS HERE===================%

    % bib stuff
    \nocite{*}
    \addtocontents{toc}{{}}
    \addcontentsline{toc}{section}{\refname}
    \bibliographystyle{plain}
    \bibliography{../Bibliography}
\end{document}

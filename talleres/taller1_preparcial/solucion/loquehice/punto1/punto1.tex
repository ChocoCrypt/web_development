% article example for classicthesis.sty
\documentclass[10pt,a4paper]{article} % KOMA-Script article scrartcl
\usepackage{import}
\usepackage{xifthen}
\usepackage{pdfpages}
\usepackage{transparent}
\newcommand{\incfig}[1]{%
    \def\svgwidth{\columnwidth}
    \import{./figures/}{#1.pdf_tex}
}
\usepackage{lipsum}     %lorem ipsum text
\usepackage{titlesec}   %Section settings
\usepackage{titling}    %Title settings
\usepackage[margin=10em]{geometry}  %Adjusting margins
\usepackage{setspace}
\usepackage{listings}
\usepackage{amsmath}    %Display equations options
\usepackage{amssymb}    %More symbols
\usepackage{xcolor}     %Color settings
\usepackage{pagecolor}
\usepackage{mdframed}
\usepackage[spanish]{babel}
\usepackage[utf8]{inputenc}
\usepackage{longtable}
\usepackage{multicol}
\usepackage{graphicx}
\graphicspath{ {./Images/} }
\setlength{\columnsep}{1cm}

% ====| color de la pagina y del fondo |==== %
\pagecolor{black}
\color{white}



\begin{document}
    %========================{TITLE}====================%
    \title{\rmfamily\normalfont\spacedallcaps{ Punto 1 taller preparcial }}
    \author{\spacedlowsmallcaps{Rodrigo Castillo}}
    \date{\today}

    \maketitle


     % ====| Loguito |==== %
    \includegraphics[width=0.1\linewidth]{negro_cara.png}
    %=======================NOTES GOES HERE===================%
    \section{enunciado}

        Un grupo de investigación en psicologı́a está interesado en conocer el efecto de la
        validación social en la toma de decisiones en problemas cotidianos. En el estudio
        se requiere un instrumento (aplicación web) para la recolección de la información,
        acontinuación, se encuentra la explicación de parte de uno de los investigadores.

        \\ Estamos estudiado la dificultad que experimentan algunas personas en la toma de decisiones frente
        a ciertos problemas cotidianos. Algunos de esos problemas no son fáciles de comunicar a los demás
        inclusive si son cercanos y esto dificulta que la persona actué oportunamente frente a los problemas.
        Queremos estudiar el efecto que tiene la interacción social en la toma de decisiones y los juicios
        colectivos frente a lo que se debe hacer en una determinada situación.

        \\ La investigación involucra un grupo de voluntarios que ha decidido participar en nuestro estu-
        dio. Le pediremos a algunas personas que “pidan un consejo” y otras deberán “aconsejarlas” para
        darle solución al problema. Para ello, requerimos una aplicación donde los investigadores puedan
        postear casos de estudio como si fuera una persona anónima y que otras personas puedan votar por
        soluciones que, en ocasiones son propuestas y en otras, los voluntarios podrán postear (votar) por
        soluciones/consejos de otras personas. Los consejos que se den deben ser lo más genuinos que se
        pueda, los participantes deben sentir eso en la aplicación.

        \\ Esta información será contrastada con los resultados de los casos de estudio, básicamente cuál
        fue el desenlace de cada situación en la vida real. De la aplicación necesitamos recopilar la informa-
        ción de cómo “aconsejaron” las personas en la red social para poder analizarlas.

        \\ Antes de comenzar las votaciones, la persona debe llenar sus datos: nombre, nombre anónimo,
        correo electrónico, género, edad. Automáticamente la aplicación debe dividir a las personas en con-
        trol y tratamiento sin que lo sepan, de esto depende los casos que se les asignen. En el grupo de
        control, se dejan solo las soluciones posibles de los casos originales que nacieron de las terapias y
        en los casos de tratamiento se permiten varios grados de interacción social

        \\Grupo I: Solo vota entre las opciones Grupo II: Vota entre las opciones y puede darle me gus-
        ta a otras Grupo III: Vota entre las opciones o puede añadir una nueva Grupo IV: Vota entre las
        opciones, puede añadir nuevas opciones y puede darle me gusta a las otras

        \\ Les dejo un ejemplo de dos casos. En este primer caso, los voluntarios solo deben escoger una
        de las soluciones propuestas, no pueden añadir nuevas opciones


        \\ Persona anónima: Hace 15 años que no hablo con mi papa, se fue con otras personas y nos dejó
        solos a mi y mi hermano, hoy se encuentra enfermo y está solo, se dé eso por otros familiares, yo
        me voy a casar pronto. ¿Deberı́a invitarlo a la boda?

        \begin{itemize}
            \item { opcion 2: No invitarloopcion 1: Invitarlo}
            \item {opcion 2: No invitarlo}
        \end{itemize}


        \\ En este segundo caso, los voluntarios pueden darle una especie de “me gusta” a todas las op-
        ciones que consideren adecuadas. Sin embargo, debe seleccionar un único consejo para avanzar al
        siguiente caso.

        \\ Persona anónima: Mi vecina vigila cada vez que alguien entra y sale de mi apartamento, me vigila
        cada vez que botó la basura y si hago algo que para ella no este bien crea un escandalo de ello. Es
        ası́ con todos los vecinos. Una vez llamó a la policı́a de tránsito porque otro vecino parqueó mal el
        carro. Es muy irritante, el fin de semana voy a hacer una fiesta en mi casa y no se cómo lidiar con
        esta persona, no quiero terminar tratándola mal.

        \begin{itemize}
            \item {Habla con ella}
            \item {Llama a la administración y avisa con anticipación}
            \item {Busca otro lugar para hacer la fiesta}
            \item {añadir opción}
        \end{itemize}

    \section{preguntas :}
        \begin{itemize}
            \item {¿Cuáles son los diferentes usuarios qué deberı́a tener la aplicación?}
            \item {Escriba las historias de usuario adecuadas para este caso}
            \item {Realice un listado de los requerimientos funcionales y no
                funcionales de la aplicación según lo manifestado por el
                investigador}
            \item {Para cada requerimiento esciba la dificultad, utilidad y
                tiempo que estime que puede llevar su respectiva elaboración}
            \item {Haga un gráfico comparativo entre la dificultad y utilidad.
                ¿Qué criterio de priorización de desarrollo de los
                requerimientos puede ser utilizado?}
            \item {Cree un mockup para la aplicación}
        \end{itemize}



    %=======================NOTES ENDS HERE===================%

    % bib stuff
    \nocite{*}
    \addtocontents{toc}{\protect\vspace{\beforebibskip}}
    \addcontentsline{toc}{section}{\refname}
    \bibliographystyle{plain}
    \bibliography{../Bibliography}
\end{document}

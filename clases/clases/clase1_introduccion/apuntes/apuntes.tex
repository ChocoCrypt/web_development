% article example for classicthesis.sty
\documentclass[10pt,a4paper]{article} % KOMA-Script article scrartcl
\usepackage{lipsum}     %lorem ipsum text
\usepackage{titlesec}   %Section settings
\usepackage{titling}    %Title settings
\usepackage[margin=10em]{geometry}  %Adjusting margins
\usepackage{setspace}
\usepackage{listings}
\usepackage{amsmath}    %Display equations options
\usepackage{amssymb}    %More symbols
\usepackage{xcolor}     %Color settings
\usepackage{pagecolor}
\usepackage{mdframed}
\usepackage[spanish]{babel}
\usepackage[utf8]{inputenc}
\usepackage{longtable}
\usepackage{multicol}
\usepackage{graphicx}
\graphicspath{ {./Images/} }
\setlength{\columnsep}{1cm}

% ====| color de la pagina y del fondo |==== %
\pagecolor{black}
\color{white}



\begin{document}
    %========================{TITLE}====================%
    \title{\rmfamily\normalfont\spacedallcaps{ Clase1 Introduccion Desarrollo web }}
    \author{\spacedlowsmallcaps{Rodrigo Castillo}}
    \date{\today} 
    
    \maketitle
     

     % ====| Loguito |==== %
    \includegraphics[width=0.1\linewidth]{negro_cara.png}
    %=======================NOTES GOES HERE===================%
    \section{introduccion}
        Como funcionan las aplicaciones web? 
        \subsection{modelo cliente servidor}
            la idea es que hay clientes que se comunican con servidores , los
            clientes mas recurrentes son computadores y los celulares  
            \\ el modelo cliente servidor funciona de la siguiente manera : 
            \\ en el servidor se almacenan todas las rutinas en con las que el
            cliente interactua , el cliente envia solicitudes al servidor y
            pidiendo información  
            \\ la forma en la que el cliente se comunica es a travez de una
            interfaz de usuario que sea interactiva y didactica y convierte
            toda la informacion que nosotros almacenamos en botones, imágenes,
            animaciones....bla bla bla  
        
    







    %=======================NOTES ENDS HERE===================%
    
    % bib stuff
    \nocite{*}
    \addtocontents{toc}{\protect\vspace{\beforebibskip}}
    \addcontentsline{toc}{section}{\refname}    
    \bibliographystyle{plain}
    \bibliography{../Bibliography}
\end{document}

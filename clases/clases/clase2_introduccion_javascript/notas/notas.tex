% article example for classicthesis.sty
\documentclass[10pt,a4paper]{article} % KOMA-Script article scrartcl
\usepackage{lipsum}     %lorem ipsum text
\usepackage{titlesec}   %Section settings
\usepackage{titling}    %Title settings
\usepackage[margin=10em]{geometry}  %Adjusting margins
\usepackage{setspace}
\usepackage{listings}
\usepackage{amsmath}    %Display equations options
\usepackage{amssymb}    %More symbols
\usepackage{xcolor}     %Color settings
\usepackage{pagecolor}
\usepackage{mdframed}
\usepackage[spanish]{babel}
\usepackage[utf8]{inputenc}
\usepackage{longtable}
\usepackage{multicol}
\usepackage{graphicx}
\graphicspath{ {./Images/} }
\setlength{\columnsep}{1cm}

% ====| color de la pagina y del fondo |==== %
\pagecolor{black}
\color{white}



\begin{document}
    %========================{TITLE}====================%
    \title{\rmfamily\normalfont\spacedallcaps{ Introduccion a JavaScript }}
    \author{\spacedlowsmallcaps{Rodrigo Castillo}}
    \date{\today} 
    
    \maketitle
     

     % ====| Loguito |==== %
    \includegraphics[width=0.1\linewidth]{negro_cara.png}
    %=======================NOTES GOES HERE===================%
    \section{intro}
        javascript es un lenguaje de alto nivel que se utiliza para el lado del
        cliente , muchos de los frameworks que hay por ahí funcionan en JS. 
    \section{Como ejecutar codigo de javascript?}
        \begin{enumerate}
            \item {El Browser} 
            \item {A travez de un script en VScode ejecutado en un archivo HTML} 
            \item {A travez de NodeJS} 
        \end{enumerate}
        
        \subsection{Ejecutar codigo JavaScript desde el Browser}
            ir a cualquier pagina desde el browser . 
            \\ inspeccionar cualquier elemento 
            \\ abrir console y corremos el comando y ya
        \subsection{Ejecutar JS a partir de un archivo HTML}
           se agrega el script y se abre con el browser
        \subsection{Ejecutar con nodeJS}
            nodejs <archivo.js>
        
        
    







    %=======================NOTES ENDS HERE===================%
    
    % bib stuff
    \nocite{*}
    \addtocontents{toc}{\protect\vspace{\beforebibskip}}
    \addcontentsline{toc}{section}{\refname}    
    \bibliographystyle{plain}
    \bibliography{../Bibliography}
\end{document}
